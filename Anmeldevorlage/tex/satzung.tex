\section{Vereinssatzung bingo e.V.}

\begin{multicols}{2}

\paragraph{Name, Sitz, Geschäftsjahr}
Der Verein führt den Namen „Bürgernetz Ingolstadt e.V.“, abgekürzt „bingo
e.V.“. Er ist in das Vereinsregister eingetragen. Der Verein hat seinen Sitz in
Ingolstadt. Das Geschäftsjahr des Vereins ist das Kalenderjahr.

\paragraph{Vereinszweck}
Zweck des Vereins ist die Förderung der Volksbildung und der beruflichen
Bildung. Der Verein wird zu diesem Zweck 
\renewcommand{\labelenumi}{\alph{enumi}}
\begin{enumerate}
\setlength{\itemsep}{-2pt}
\item interessierte Bevölkerungskreise durch geeignete Veranstaltungen und Veröffentlichungen an das Bürgernetz
heranführen, 
\item hierzu Fortbildungsveranstaltungen und Seminare
durchführen und geeignetes Lehrmaterial erstellen und abgeben, 
\item mit steuerbegünstigten Einrichtungen zusammenarbeiten, soweit diese
vergleichbare Zwecke verfolgen.
\end{enumerate}
%\renewcommand{\labelenumi}{\theenumi)}

\paragraph{Gemeinnützigkeit}
Der Verein verfolgt ausschließlich und unmittelbar gemeinnützige Zwecke
im Sinne des Abschnitts „Steuerbegünstigte Zwecke“ der Abgabenordnung.
Der Verein ist selbstlos tätig; er verfolgt nicht in erster Linie
eigenwirtschaftliche Zwecke. Mittel des Vereins dürfen nur für die
satzungsgemäßen Zwecke verwendet werden. Die Mitglieder erhalten keine
Zuwendungen aus Mitteln des Vereins. Es darf keine Person durch
Ausgaben, die dem Zweck der Körperschaft fremd sind, oder durch
unverhältnismäßig hohe Vergütungen begünstigt werden.

\paragraph{Mitglieder}
Mitglied kann jede natürliche und juristische Person werden. Der Antrag auf
Aufnahme in den Verein ist schriftlich beim Vorstand einzureichen. Über die
Aufnahme entscheidet der Vorstand. Neben ordentlichen Mitgliedern gibt es
Ehrenmitglieder. Die Ehrenmitgliedschaft ist beitragsfrei. Der Vorstand
verleiht die Ehrenmitgliedschaft. Die Mitgliedschaft endet mit dem Tod,
durch Austritt, durch Streichung oder durch Ausschluss aus dem Verein.
Der Austritt ist schriftlich gegenüber dem Vorstand zu erklären. Er ist zum
Ende eines Kalenderjahres unter Einhaltung einer Kündigungsfrist von drei
Monaten zulässig. Ein Mitglied kann, wenn es gegen die Vereinsinteressen
gröblich verstoßen hat, durch Beschluss des Vorstandes aus dem Verein
ausgeschlossen werden. Vor der Beschlussfassung ist dem Mitglied unter
Setzung einer angemessenen Frist Gelegenheit zu geben, sich persönlich
vor dem Vorstand oder schriftlich zu rechtfertigen. Der Beschluss über den
Ausschluss ist mit Gründen zu versehen und dem Mitglied mittels
eingeschriebenen Briefes bekannt zu geben. Gegen den
Ausschließungsbeschluss des Vorstandes steht dem Mitglied das Recht der
Berufung an die nächste Mitgliederversammlung zu. Die Berufung muss
innerhalb einer Frist von einem Monat ab Zugang des
Ausschließungsbeschlusses beim Vorstand eingelegt werden. Macht das
Mitglied von dem Recht der Berufung gegen den Ausschließungsbeschluss
keinen Gebrauch oder versäumt es die Berufungsfrist, so unterwirft es sich
damit dem Ausschließungsbeschluss mit der Folge, dass der Ausschluss
gerichtlich nicht angefochten werden kann. Während des
Ausschlussverfahrens ruhen alle Ämter des betreffenden Mitglieds.
Ausgeschiedene Mitglieder haben aus ihrer Mitgliedschaft keinen Anspruch
an das Vereinsvermögen. Mit dem Ende der Mitgliedschaft enden alle
Ämter und Aufgaben des gewesenen Mitglieds ohne besonderes Verfahren.
\paragraph{Mitgliedsbeiträge}
Von den Mitgliedern wird ein Jahresbeitrag erhoben, dessen Höhe die
Mitgliederversammlung festsetzt.
\paragraph{Organe des Verein}
Organe des Vereins sind der Vorstand und die Mitgliederversammlung.
\paragraph{Vorstand}
Der Vorstand besteht aus dem Vorsitzenden, dem stellvertretenden
Vorsitzenden, dem Schriftführer und dem Kassenwart. Der Vorstand wird
von der Mitgliederversammlung auf 2 Jahre gewählt. Die
Vorstandsmitglieder bleiben auch nach dem Ablauf ihrer Amtszeit bis zur
Neuwahl im Amt. Die Mitglieder des Vorstandes sind von den
Beschränkungen des § 181 BGB befreit, soweit sie als Vorstandsmitglieder
mit sich selbst als Vertreter einer juristischen Person Rechtsgeschäfte
vornehmen. Soweit Vorstandsmitglieder im eigenen Namen oder als
Vertreter natürlicher Personen mit dem Verein Rechtsgeschäfte vornehmen
wollen, sind sie an dessen Vertretung gehindert. Der Vorstand entscheidet
dann ohne Zuziehung der gehinderten Mitglieder.
\paragraph{Zuständigkeit des Vorstands}
Der Vorstand ist für alle Angelegenheiten des Vereins zuständig, die nicht
durch diese Satzung anderen Vereinsorganen vorbehalten sind.
Er hat vor allem folgende Aufgaben:

\begin{enumerate}
\setlength{\itemsep}{-2pt}
\item Vorbereitung der Mitgliederversammlungen und Aufstellung der
Tagesordnung,
\item Einberufung der Mitgliederversammlung,
\item Vollzug der Beschlüsse der Mitgliederversammlung,
\item Verwaltung des Vereinsvermögens,
\item Erstellung des Jahres- und Kassenberichts,
\item Beschlussfassung über die Aufnahme und den Ausschluss von Vereinsmitgliedern.
\end{enumerate}

Der Vorsitzende oder der stellvertretende Vorsitzende vertritt zusammen
mit einem weiteren Mitglied den Verein gerichtlich und außergerichtlich.
Rechtsgeschäfte mit einem Betrag über \EUR{500,-}  sind für den Verein nur
verbindlich, wenn der Vorstand zugestimmt hat.

\paragraph{Sitzung des Vorstands}
Für die Sitzung des Vorstands sind die Mitglieder vom Vorsitzenden, bei
seiner Verhinderung vom stellvertretenden Vorsitzenden, rechtzeitig, jedoch
mindestens eine Woche vorher, einzuladen. Der Vorstand ist
beschlussfähig, wenn mindestens drei Mitglieder anwesend sind. Der
Vorstand entscheidet mit einfacher Mehrheit der abgegebenen gültigen
Stimmen. Bei Stimmengleichheit entscheidet die Stimme des Vorsitzenden
bzw. des die Sitzung leitenden Vorstandsmitglieds. Über die Sitzung des
Vorstands ist vom Schriftführer ein Protokoll aufzunehmen. Die
Niederschrift soll Ort und Zeit der Vorstandssitzung, die Namen der
Teilnehmer, die Beschlüsse und das Abstimmungsergebnis enthalten.

\paragraph{Kassenführung}
Die zur Erreichung des Vereinszwecks erforderlichen Mittel werden in
erster Linie aus Beiträgen und Spenden aufgebracht. Der Kassenwart hat
über die Kassengeschäfte Buch zu führen und eine Jahresrechnung zu
erstellen. Zahlungen dürfen nur auf Grund von Auszahlungsanordnungen
eines Mitglieds des Vorstands geleistet werden. Die Jahresrechnung ist von
zwei Kassenprüfern, die jeweils auf 2 Jahre gewählt werden, zu prüfen. Sie
ist der Mitgliederversammlung zur Genehmigung vorzulegen.

\paragraph{Mitgliederversammlung}
Die Mitgliederversammlung ist für folgende Angelegenheiten zuständig:
\begin{enumerate}
\setlength{\itemsep}{-2pt}
\item Entgegennahme der Berichte des Vorstands,
\item Festsetzung der Höhe des Jahresbeitrags,
\item Wahl und Abberufung der Vorstandsmitglieder und der Kassenprüfer,
\item Beschlussfassung über die Geschäftsordnung für den Vorstand,
\item Beschlussfassung über Änderungen der Satzung und über die Auflösung
des Vereins,
\item Beschlussfassung über die Berufung gegen einen Beschluss des
Vorstandes über einen abgelehnten Aufnahmeantrag und über einen
Ausschluss.
\end{enumerate}
Die ordentliche Mitgliederversammlung findet jährlich mindestens einmal
statt. Außerdem muss die Mitgliederversammlung einberufen werden, wenn
das Interesse des Vereins es erfordert oder wenn die Einberufung von
einem Fünftel der Mitglieder unter Angabe des Zwecks und der Gründe
vom Vorstand schriftlich verlangt wird. Jede Mitgliederversammlung wird
vom Vorsitzenden, bei seiner Verhinderung vom stellvertretenden
Vorsitzenden, unter Einhaltung einer Frist von zwei Wochen durch
persönliches Einladungsschreiben einberufen. Dabei ist die vorgesehene
Tagesordnung mitzuteilen. Jedes Mitglied kann bis spätestens eine Woche
vor dem Tag der Mitgliederversammlung beim Vorsitzenden schriftlich
beantragen, dass weitere Angelegenheiten nachträglich auf die
Tagesordnung gesetzt werden. Über Anträge auf Ergänzung der
Tagesordnung, die erst in der Versammlung gestellt werden, beschließt die
Mitgliederversammlung. Satzungsänderungen oder die Auflösung des
Vereins können nicht per Dringlichkeit beantragt werden.

\paragraph{Beschlussfassung der Mitgliederversammlung}
Die Mitgliederversammlung wird vom Vorsitzenden, bei dessen
Verhinderung vom stellvertretenden Vorsitzenden oder einem anderen
Vorstandsmitglied geleitet. Bei Wahlen kann die Versammlungsleitung für
die Dauer des Wahlgangs und der vorhergehenden Aussprache einem
Wahlausschuss übertragen werden. In der Mitgliederversammlung ist jedes
Mitglied stimmberechtigt. Beschlussfähig ist jede ordnungsgemäß
einberufene Mitgliederversammlung. Soweit die Satzung nichts anderes
bestimmt, entscheidet bei der Beschlussfassung die einfache Mehrheit der
abgegebenen Stimmen; Stimmenthaltungen bleiben außer Betracht. Zur
Änderung der Satzung und zur Auflösung des Vereins ist eine Mehrheit von
drei Viertel der abgegebenen Stimmen erforderlich. Die Art der Abstimmung
wird grundsätzlich vom Vorsitzenden als Versammlungsleiter festgesetzt.
Die Abstimmung muss jedoch geheim durchgeführt werden, wenn ein
Fünftel der erschienenen Mitglieder dies beantragt.
Über den Verlauf der Mitgliederversammlung ist ein Protokoll aufzunehmen,
das vom Vorsitzenden zu unterzeichnen ist. Die Niederschrift soll Ort und
Zeit der Versammlung, die Zahl der erschienenen Mitglieder, die Person
des Versammlungsleiters, die Tagesordnung, die Beschlüsse, die
Abstimmungsergebnisse und die Art der Abstimmung enthalten.

\paragraph{Auflösung}
Die Auflösung des Vereins kann nur in einer zu diesem Zweck einberufenen
Mitgliederversammlung beschlossen werden. Bei Auflösung des Vereins
oder bei Wegfall steuerbegünstigter Zwecke fällt das Vermögen des
Vereins an die Stadt Ingolstadt, die es unmittelbar und ausschließlich für
gemeinnützige Zwecke im Sinne dieser Satzung zu verwenden hat.

\end{multicols}

\vfill
--- \emph{Die vorstehende Satzung wurde in der Mitgliederversammlung vom
29.04.2003 beschlossen.}
\clearpage
