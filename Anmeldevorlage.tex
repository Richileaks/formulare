\documentclass[a4paper]{scrartcl}
\usepackage{ngerman}
%\usepackage[latin1]{inputenc}
\usepackage[utf8]{inputenc}
\usepackage[T1]{fontenc}
\usepackage[pdftex]{hyperref}
\usepackage{fancyhdr} 
\usepackage{german,latexsym,textcomp}
\usepackage{tabularx}
\usepackage{multirow}
\usepackage[pdftex]{graphicx}
\usepackage{framed}
\usepackage{wasysym}
\usepackage{fancyhdr}
\pagestyle{fancy}
\fancyhf{}

\fancyfoot[R]{\thepage}
\usepackage{geometry}
\geometry{
  left=3cm,
  right=3cm,
  top=2cm,
  bottom=4cm,
  bindingoffset=5mm
}
\def\LayoutTextField#1#2{% label, field
  \leavevmode#2\vskip0.05em 
	 #1%
}
\def\LayoutCheckBox#1#2{% label, field
  #1#2
}
\setlength{\parindent}{0in} 
%\usepackage[pdftex]{rotating}
%----------------------------------------------------
\newcommand{\KeinLabel}{\ \hspace*{-0.17cm}}
%---Rahmenfarben-------------------------------------
\newcommand{\Weiss}{1 1 1}
\newcommand{\Grau}{0.8 0.8 0.8}
\newcommand{\Schwarz}{0 0 0}
\newcommand{\Gruen}{0 1 0}
\newcommand{\Rot}{1 0 0}
\newcommand{\Blau}{0 0 1}
\newcommand{\Gelb}{1 1 0}
%--------------------------------------------------------------------

\begin{document}


\newgeometry{
  left=3cm,
  right=3cm,
  top=1cm,
  bottom=4cm,
  bindingoffset=5mm
}

\begin{sffamily}
\thispagestyle{empty}

\begin{minipage}[b]{0.3\textwidth}
\begin{flushleft}
Bürgernetzverein Ingolstadt\\
bingo e.V.\\
Krumenauer Str. 54\\
\vskip1em
85049 Ingolstadt\\
\end{flushleft}
\end{minipage}
\hspace{\fill}%
\begin{minipage}[b]{0.6\textwidth}
\begin{flushright}
\includegraphics[width=\textwidth]{bingo.png}\\
\tiny{Version 05/2015 vom 23.04.2015}
\end{flushright}
\end{minipage}
\vskip4em 

\textbf{Antrag auf Mitgliedschaft bei bingo e.V.}\\


\TextField[name=FIRMA,width=\textwidth,bordercolor=\Schwarz, borderstyle=U]{(Firma/Verein)}\\
\vskip1em
\TextField[name=ANREDE,width=\textwidth,bordercolor=\Schwarz, borderstyle=U]{(Anrede und ggf. Titel)}\\
\vskip1em
\TextField[name=NAME,width=\textwidth,bordercolor=\Schwarz, borderstyle=U]{(Name, Vorname)}\\
\vskip1em
\TextField[name=STR,width=\textwidth,bordercolor=\Schwarz, borderstyle=U]{(Straße)}\\
\vskip1em
\TextField[name=PLZ,width=\textwidth,bordercolor=\Schwarz, borderstyle=U]{(PLZ, Ort)}\\
\vskip1em
\TextField[name=KLASSE,width=\textwidth,bordercolor=\Schwarz, borderstyle=U]{(Beitragsklasse)}\\
\vskip1em
\TextField[name=GEB, width=\textwidth,bordercolor=\Schwarz, borderstyle=U]{(Geburtsdatum)}\\
\vskip1em

\vskip4em 
\begin{framed}\begin{itemize}
\item[\huge\Square] Vorname, Name und E-Mailadresse sollen 
nicht in öffentliche Verzeichnisse aufgenommen werden.
\end{itemize}
\end{framed}
\begin{framed}{\textbf{Hinweise zum Vereinsbeitrag:}\\
Der bingo e.V. Vereinsbeitrag ist steuerbegünstigt, siehe Seite 3. }
\end{framed}
\begin{framed}
Bürgernetzverein Ingolstadt, abgekürzt bingo e.V.\\
Krumenauer Straße 54\\
85049 Ingolstadt\\
Gläubiger-Identifikationsnummer: 
\vskip1em
\begin{center} \large SEPA-Lastschriftmandat \end{center}

Mandatsreferenz wird seperat mitgeteilt
\vskip1em
Ich ermächtige den Zahlungsempfänger bingo e.V. Zahlungen von meinem Konto mittels Lastschrift einzuziehen. Zugleich weise ich mein Kreditinstitut an, die vom Zahlungsempfänger bingo e.V. auf mein Konto gezogenen Lastschriften einzulösen.  
\vskip1em
\textbf{Hinweis: Ich kann innerhalb von acht Wochen, beginnend mit dem Belastungsdatum, die Erstattung des belasteten Betrages verlangen. Es gelten dabei die mit meinem Kreditinstitut vereinbarten Bedingungen.}
\vskip1em
\textbf{Zahlungsart: Wiederkehrende Zahlung}
\end{framed}
%------------------------------------------------------------------------------------------------------------
% HINWEISE
%------------------------------------------------------------------------------------------------------------
% Es folgen 3 Optionen, ob und wie die Formulardaten verarbeitet werden koennen.
% Es darf nur EINE \begin{form}-Zeile fuer (1),(2) oder (3) ohne Kommentarzeichen "%" verwenden werden.
% fuer (1) ein sinnvolle e-Mail-Adresse eingeben
% fuer (2) die korrekte URL fuer das Server-Skript einsetzen, das die Daten verarbeiten soll
% fuer (3) den Button [Senden] (\Submit{Senden}) unten im Formular entfernen (NUR Ausdruck)
%------------------------------------------------------------------------------------------------------------

%------------------------------------------------------------------------------------------------------------
%--(1)-------------------------------------------------------------------------------------------------------
%---Formularinhalte als Mail versenden, wenn Benutzer auf [Senden] drueckt-----------------------------------
%------------------------------------------------------------------------------------------------------------
 %------------------------------------------------------------------------------------------------------------

%------------------------------------------------------------------------------------------------------------
%--(2)-------------------------------------------------------------------------------------------------------
%---Formularinhalte an einen Server versenden, wenn Benutzer auf [Senden] drueckt----------------------------
%------------------------------------------------------------------------------------------------------------
% \begin{Form}[action=http://www.meinserver.com/cgi-bin/verarbeiteformular.cgi,encoding=html,method=get]
%------------------------------------------------------------------------------------------------------------

%------------------------------------------------------------------------------------------------------------
%--(3)-------------------------------------------------------------------------------------------------------
%---Formularinhalte nur zum Ausfuellen und Ausdrucken. Button [Senden] (\Submit{Senden}) entfernen-----------
%------------------------------------------------------------------------------------------------------------
% \begin{Form}
%------------------------------------------------------------------------------------------------------------


\end{sffamily}
\end{document}
