\documentclass{bingoevletter}
\usepackage[ngerman]{babel}
\usepackage{blindtext}
\usepackage[T1]{fontenc}
\usepackage[utf8]{inputenc}
\usepackage[table]{xcolor}
\usepackage{tabu}
\usepackage{framed}
\usepackage{eurosym}
\usepackage{multicol}
\usepackage{rotating}
\usepackage{tabularx}


\usepackage{hyperref}
\hypersetup{colorlinks,linkcolor=blue}

\renewcommand{\DefaultOptionsofCheckBox}{print,bordercolor=black}
\renewcommand{\DefaultOptionsofText}{print,bordercolor=black,borderstyle=U}
\renewcommand{\DefaultOptionsofComboBox}{print,sort,bordercolor=black,borderstyle=U}
\newcolumntype{C}[1]{>{\centering\arraybackslash}p{#1}}
\setlength{\parindent}{0pt}



\newcommand{\Anrede}{Frau}
\newcommand{\Vorname}{Erika}
\newcommand{\Nachname}{Mustermann}
\newcommand{\Adresse}{Heidestr. 17}
\newcommand{\PLZ}{12345}
\newcommand{\Ort}{Berlin}
\newcommand{\EingangDatum}{10.12.2018}
\newcommand{\AustrittJahr}{2019}
\newcommand{\Login}{EM1337}




\begin{document}
\begin{letter}{% 
		\BehoerdeName\\
		\BehoerdeStr\\
		\BehoerdeOrt%
	}
	\opening{Ausstellung eines Erweiterten Führungszeugnisses für die Ehrenamtliche und Unentgeltliche Tätigkeit.\\ Antrag auf Gebührenbefreiung.}
	
	Sehr geehrte Damen und Herren,
	
	
	
	Herr/Frau \Vorname\ \Nachname , geb. am \GeburtsDatum\
	wohnhaft \Adresse,\ \PLZ\ \Ort\
	ist bei uns als Dozent ehrenamtlich und unentgeltlich tätig.\\
	
	
	Im Rahmen seines/ihres Ehrenamts u.a. die Beaufsichtigung Betreuung, Erziehung und Ausbildung von Minderjährigen (§30a Abs. 1 Nr. 2b BZRG) zu seinem/ihrem Aufgabenbereich.\\
	
	Aus diesem Grund bitten wir mit Bezug auf das Bundeszentralregistergesetz (§30a) um Erstellung eines erweiterten Führungszeugnisses zur Vorlage bei uns.\\
	Unter Hinweis auf die Richtlinien des Bundesamtes für Justiz beantragen wir zugleich die Gebührenfreiheit.\\
	Wir bedanken uns für Ihre Unterstützung und verbleiben\\
	
	
		
	
	\closing{mit freundlichen Grüßen}
\end{letter}	
\end{document}